\documentclass[abstract=off,10pt,a4paper,bibliography=totocnumbered]{article}
\usepackage[paper=a4paper,left=35mm,right=35mm,top=25mm,bottom=30mm]{geometry}
\usepackage[doublespacing]{setspace}
\usepackage[english]{babel}
\usepackage[utf8]{inputenc}
\usepackage[round]{natbib}
\usepackage{amsmath}
\usepackage{colortbl}
\usepackage{amsfonts}
\usepackage{amssymb}
\usepackage{gensymb}
\usepackage{graphicx}
\usepackage{tikz}
\usepackage{enumerate}
\usepackage{enumitem}
\usepackage{subcaption}
\usepackage{booktabs}
\usepackage[hidelinks]{hyperref}
\usepackage[nameinlink]{cleveref}
\usepackage{lineno}
\usepackage{multirow}
\usepackage{arydshln}
\usepackage[flushleft]{threeparttable}

%------------------------------------------------------------------------------
%	Some Styling
%------------------------------------------------------------------------------
% Creating some TikZ styles
\tikzset{
  nonterminal/.style = {rectangle
    , minimum size = 6mm
    , very thick
    , draw = black!
  }
}

% Changing the style of captions in figures etc.
\captionsetup{labelfont=bf, format=plain, font=small}

% For supplementary material
\newcommand{\beginappendix}{%
  \setcounter{table}{0}
  \renewcommand{\thetable}{S\arabic{table}}%
  \setcounter{figure}{0}
  \renewcommand{\thefigure}{S\arabic{figure}}%
  \setcounter{equation}{0}
  \renewcommand{\theequation}{Equation S\arabic{equation}}%
  \setcounter{section}{0}
  \renewcommand{\thesection}{A.\arabic{section}}%
}

%------------------------------------------------------------------------------
%	Titlepage: Header
%------------------------------------------------------------------------------
\title{\textbf{Appendix}\\ Bound within Boundaries: How Well Do Protected Areas
Match Movement Corridors of Their Most Mobile Protected Species?}

% List of Authors
\author{
  David D. Hofmann\textsuperscript{1,\S,*} \and
  Dominik M. Behr\textsuperscript{1,2,*} \and
  John W. McNutt\textsuperscript{2} \and
  Arpat Ozgul\textsuperscript{1} \and
  Gabriele Cozzi\textsuperscript{1,2}
}

% Reduce spacing between authors
\makeatletter
\def\and{%
  \end{tabular}%
  \hskip -0.5em \@plus.17fil\relax
  \begin{tabular}[t]{c}}
\makeatother

% Current Date
\date{\today}

% And here the masterpiece begins
\begin{document}

% Change page numbering
\pagenumbering{gobble}

% Required to be able to cite
\bibliographystyle{apalike}

% Create Titlepage
\maketitle

%------------------------------------------------------------------------------
%	Titlepage: Additional Info
%------------------------------------------------------------------------------
\begin{flushleft}

\vspace{0.5cm}

\textsuperscript{1} Department of Evolutionary Biology and Environmental
Studies, University of Zurich, Winterthurerstarsse 190, 8057 Zurich,
Switzerland.

\textsuperscript{2} Botswana Predator Conservation Trust, Private Bag 13, Maun,
Botswana.

\textsuperscript{\S} Corresponding author (david.hofmann2@uzh.ch)

\textsuperscript{*} Shared first authorship

\vspace{4cm}

\textbf{Running Title:} Connectivity across a Transfrontier Conservation Area.

\vspace{0.5cm}

\textbf{Keywords:} dispersal, habitat selection, integrated step selection
function,Kavango-Zambezi Transfrontier Conservation Area, landscape
connectivity, least-cost corridors, Lycaon pictus, permeability surface,
protected areas, wildlife management

\end{flushleft}

%------------------------------------------------------------------------------
%	Main Text
%------------------------------------------------------------------------------
\newpage

% Change page numbering
\pagenumbering{arabic}

% Create linenumbers
\linenumbers

% Change to appendix counters
\appendix
\beginappendix

%------------------------------------------------------------------------------
%	Appendix S1: Net Squared Displacement
%------------------------------------------------------------------------------
\section{Net Squared Displacement}
\begin{figure}[hbpt]
  \begin{center}
    \includegraphics[width = \textwidth]{99_NSD.pdf}
    \caption{NSD displacement through time for one of our dispersers. The blue
    line indicates the period during which we calssified the individual as
    dispersing.}
    \label{NSD}
  \end{center}
\end{figure}

%------------------------------------------------------------------------------
%	Appendix S2: GPS Data
%------------------------------------------------------------------------------
\newpage
\section{GPS Data}
\begin{figure}[hbtp]
  \begin{center}
    \includegraphics[width = 0.86\textwidth]{99_Trajectories.pdf}
    \caption{Illustration of all trajectories that we recorded. Each color
    represents a different dispersing coalition. All coalitions departed from
    the core study area, which is indicated by the white rectangle. The
    coalition dispersing towards the far east of the map covered over 360 km in
    under 10 days. Satellite background imagery was provided by Microsoft Bing.}
    \label{Trajectories}
  \end{center}
\end{figure}

\begin{table}[hbtp]
  \caption{Summary statistics of all GPS relocations that have been recorded on
  dispersing coalitions.}
  \label{GPSData}
  \begin{center}
    \resizebox{0.9\textwidth}{!}{
      \begin{tabular}{lccccccc}
      \toprule
      \parbox[]{2cm}{\centering Coalition ID} &
        \parbox[]{1cm}{\centering Sex} &
          \parbox[]{2cm}{\centering \ Pack \\ Affiliation} &
            \parbox[]{1.5cm}{\centering \# Fixes \\ Total} &
              \parbox[]{1.5cm}{\centering \# Fixes \\ During \\ Dispersal} &
                \parbox[]{1.5cm}{\centering \# Days \\ Dispersing} &
                  \parbox[]{2cm}{\centering Euclidean \\ Dispersal\\ Distance \\ (in km)} &
                    \parbox[]{2cm}{\centering Cumulative \\ Dispersal \\ Distance \\ (in km)} \\
      \midrule
      Amacuro & F & MB & 954 & 583 & 137 & 23 & 1'090 \\
      Belgium & M & ZU & 1'097 & 158 & 28 & 18 & 319 \\
      Dalwhinnie & F & PA & 545 & 62 & 22 & 50 & 243 \\
      Denali & F & MN & 1'096 & 173 & 33 & 11 & 528 \\
      Everest & M & MN & 389 & 123 & 38 & 67 & 572 \\
      Kalahari & F & HT & 1'753 & 467 & 130 & 20 & 1'963 \\
      Karisimbi & M & MN & 438 & 141 & 34 & 45 & 251 \\
      Liuwa & F & AP & 946 & 92 & 19 & 144 & 451 \\
      Lupe & M & KW & 2'209 & 396 & 34 & 8 & 436 \\
      MadameChing & F & AP & 776 & 729 & 136 & 263 & 1'560 \\
      Mirage & M & HT & 814 & 182 & 36 & 7 & 435 \\
      Odzala & M & AP & 1'410 & 205 & 42 & 53 & 412 \\
      Scorpion & M & KB & 2'676 & 393 & 34 & 4 & 471 \\
      Stetson & M & MT & 384 & 383 & 33 & 3 & 481 \\
      Taryn & F & AP & 896 & 37 & 9 & 10 & 130 \\
      \bottomrule
      \end{tabular}
    }
  \end{center}
\end{table}

%------------------------------------------------------------------------------
%	Appendix S3: Spatial Covariates
%------------------------------------------------------------------------------
\newpage
\section{Spatial Covariates}
\subsection{Land Cover}
We divided land cover into water and dryland. Water included rivers, wetlands,
and swamps. Because the inundation extent of the flood in the Okavango Delta is
highly variable within and between years, we created dynamic ``flood maps'' that
were updated every 8\textsuperscript{th} day following a remote sensing
algorithm developed by \cite{Wolski.2017}. The algorithm uses thresholding of
short wavelength infrared reflectances of MODIS Terra satellite imagery
(MCD43A4; \citealp{Schaaf.2015}) to distinguish areas covered by water and
dryland (details in Appendix A.3). While we created dynamic flood maps for the
Okavango Delta, we assumed the extent of all other water bodies (e.g. Chobe
river, Zambezi river) to be static within and between years. This static
representation was based on Globeland's land cover dataset \citep{Chen.2015},
from which we only retained the categories \textit{wetland} and \textit{water
bodies} and collectively reclassified them to \textit{water}. Globeland had an
original resolution of 30m x 30m, so we coarsened the layer to 250m x 250m using
the mode of each 250m x 250m cell. We further improved river representation by
employing the rasterized MERIT Hydro dataset \citep{Yamazaki.2019} from which we
added all rivers with a width of over 10m to our Globeland layer. We merged
dynamic and static water maps into a large rasterstack, covering the entire
study area. We also created a rasterstack rendering the covariate
\textit{distance to water} by calculating the eculidean distance of each raster
cell in the study area to the nearest source of water.

We further subdivided dryland into three layers as derived from the MODIS Terra
Vegetation Continuous Fields dataset (MOD44B; \citealp{Dimiceli.2015}). The
three layers depicted percentage cover of tree-vegetation (henceforth
\textit{trees}), non-tree-vegetation (henceforth \textit{shrubs/grassland}), and
non-vegetated (henceforth \textit{bare land}) and added up to 100\% of dryland
coverage. We used our flood map that aligned with the creation date of these
MODIS layers and defined anything covered by water as 0\% vegetated. The MODIS
vegetation layers had a resolution of 250m x 250m and no coarsening or
interpolation was required.

\subsection{Protection Status}
We created a binary layer separating protected from unprotected land. We
downloaded corresponding data on protection status in shapefile format from the
Peace Parks Foundation (\url{www.peaceparks.org}; \citealp{PeaceParks.2019}).
Protected areas included forest reserves, game reserves, wildlife management
areas, and national parks. We classified anything not covered by these
categories as unprotected (e.g. communal pastoral land, private land). We
rasterized the two categories to the binary raster \textit{protection status} (1
= protected, 0 = pastoral) with a resolution of 250m x 250m.

\subsection{Human Influence}
We created a raster layer representing human influence by integrating
information on (1) human density, (2) farming, and (3) roads.

\begin{enumerate}[label = (\arabic*)]

  \item We obtained spatial human density estimates through a publicly available
  30m x 30m high-resolution population density dataset
  (\url{www.dataforgood.fb.com}; \citealp{Facebook.2019}). We coarsened the
  layer to 250m x 250m by summing up human density values within each 250m x
  250m cell.

  \item We sourced spatial information on farms from the Globeland
  \citep{Chen.2015} and Cropland \citep{Xiong.2017} land cover datasets from
  which we retained areas that were classified as either \textit{cultivated
  land} or \textit{croplands}. Any other land cover class was not pertinent to
  farming and therefore omitted. Because both layers had a resolution of 30m x
  30m we coarsened them to 250m x 250m by assigning a value of 1 to any 250m x
  250m cell that covered farmland and a value 0 otherwise. Thus, the final layer
  depicted presence (= 1) or absence (= 0) of farms within each 250m x 250m
  cell.

  \item We obtained geo-referenced data on roads from Open Street Map
  \citep{OpenStreetMap.2019}, downloaded through Geofabrik
  (\url{www.geofabrik.de}). We only retained main tarmac roads and omitted
  smaller roads (Table S2) as these are scarcely frequented and do not represent
  an obstacle to wild dog movements \citep{Abrahms.2016}. We rasterized main
  tarmac roads to the binary raster \textit{roads} (1 = roads, 0 = no roads)
  with 250m x 250m resolution. Finally, we created the covariate
  \textit{distance to roads} by calculating the eculidean distance of each
  raster cell in the study area to the nearest road.

\end{enumerate}

\noindent Because layers (1), (2), and (3) depicted features that are typically
spatially clustered and because not all dispersing coalitions moved within
meaningful distance to each of these features, we totaled the layers
\textit{human density}, \textit{farming}, and \textit{roads}. This resulted in a
single covariate called \textit{human influence}. Note that totaling the layers
implied that the three covariates received different weights (see Appendix A.6
for details). To reduce the influence of outliers, totaled values were limited
to a maximum of 50, which visually resulted in a good balance between high and
low anthropogenic influence and was therefore considered appropriate for our
analysis. To render the fact that humans influence their surroundings beyond
their presence, we followed \cite{Elliot.2014} and applied to each raster cell a
5 km focal buffer within which we summed up and log-transformed human-influence
values (Figure S6).

%------------------------------------------------------------------------------
%	Appendix S3: Floodmapping Algorithm
%------------------------------------------------------------------------------
\newpage
\section{Flood-Mapping Algorithm}
To implement the flood mapping algorithm, we defined two sets of polygons
located in the region of the Okavango Delta (\Cref{Floodmapping}). The first set
consisted of areas known to be permanent dryland, whereas the second set
consisted of permanent waters. Since we were unable to retrieve the original
polygons used in \cite{Wolski.2017}, we geo-referenced and digitized the
polygons reported in their publication. After recreating the polygons, we used
the R-package \textit{getSpatialData} \citep{Schwalb.2018} to download and
pre-process all relatively cloud-free MODIS Terra images (MCD43A4;
\citealp{Schaaf.2015}) available for the period of our dispersal events.
Assessment of cloud cover was based on visual inspection of MODIS images on
ORI's website \citep{ORI.2019}. The MCD43A4 MODIS product is particularly useful
for dynamic flood mapping because it is updated with new composite images every
8\textsuperscript{th} day \citep{Wolski.2017}. After download, we classified
each MODIS image into a binary map of water (flood) and dryland using a
threshold that was identified as follows. First, we extracted all reflectance
values of MODIS Terra Band 7 within the water- and dryland-polygons. Second, we
computed histograms of water-reflectances and dryland-reflectances and
empirically verified that reflectances of the two groups were sufficiently
distinct. More specifically, we checked if superimposing the two histograms
resulted in a bimodal plot. This was said to be achieved if the
99\textsuperscript{th} percentile of water-reflectances did not severely exceed
the 1\textsuperscript{st} percentile of dryland-reflectances (\(p_{0.99, water}
- \frac{10}{255} < p_{0.01, dryland}\)). Third, if peaks were sufficiently
different, we calculated a threshold (\(t\)) using \ref{EQ1}:

\begin{equation}
\label{EQ1}
t = \widetilde{p}_{water} + 0.3 * (\widetilde{p}_{dryland} -
\widetilde{p}_{water})
\end{equation}

\noindent where \(\widetilde{p}_{water}\) and \(\widetilde{p}_{dryland}\) were
the median reflectances of water and dryland, respectively. We then classified
all pixels of MODIS Terra Band 7 with a value greater than \(t\) as dryland and
all pixels with a value smaller than \(t\) as water.

\begin{figure}[hbtp]
  \begin{center}
    \includegraphics[width = \textwidth]{99_Floodmapping.pdf}
    \caption{Images describing the flood mapping algorithm. (a) The colored
    polygons indicate permanent waters and permanent dryland. Below these
    polygons we extracted reflectance values of MODIS Terra Band 7 and used
    their repsective medians to calculate a classification threshold. (b)
    Example of a classified MODIS Terra Band 7 image after application of the
    threshold.}
    \label{Floodmapping}
  \end{center}
\end{figure}

% Dynamic Wetmask
\noindent Importantly, bimodality was not always achieved and in some cases no
flood map could be calculated. In fact, it appears that non-bimodality caused
the ORI algorithm to fail since the end of 2018, such that no flood maps have
been generated since then (ORI, personal comm.). We hypothesized that this was
caused by the application of static water-polygons that did not cover permanent
waters correctly anymore. Therefore, we revised the algorithm and allowed for a
more dynamic polygonization of water. That is, for each MODIS image we
calculated new water-polygons comprising areas that were covered by the flood in
99\% of the flood maps from the previous five years. All of the necessary
flood maps from previous years were kindly provided to us by ORI. Using this
slightly amended approach, we were able to address some of the bimodality issues
and to classify several additional flood maps for the period of our study.
Because MODIS Terra Band 7 had a resolution of 500m x 500m, we interpolated all
maps to 250m x 250m.

% Validation
To validate and compare the performance of our own algorithm to the original
ORI-algorithm, we randomly sampled 48 dates for which ORI prepared classified
images. To make sure that months were equally represented in the sampled dates,
we employed stratified sampling based on months (regardless of the year) and
randomly sampled four maps for each month. For the sampled dates we downloaded
and classified MODIS Terra Band 7 images and compared our classified images to
those provided by ORI (\Cref{FloodmappingValidation}). For each pair of maps we
created a difference map indicating false positives and false negatives and
computed the relative number of wrongly classified pixels. We achieved an
overall accuracy of 97\%, which presumably is an underestimate of the true
performance, as we introduced some errors when resampling the ORI-maps to our
reference raster.

\begin{figure}[hbtp]
  \begin{center}
    \includegraphics[width = \textwidth]{99_FloodmappingValidation.pdf}
    \caption{Validation procedure of our flood mapping algorithm. (a) Classified
    image that was provided to us by ORI. (b) Image for the same date but now
    classified using our own algorithm. (c) Difference image indicating false
    positives and false negatives in our own classification.}
    \label{FloodmappingValidation}
  \end{center}
\end{figure}

% Static Watermap
\noindent While water in the Okavango Delta was mapped dynamically, we used a
static watermap for its surroundings. This static map was based on Globeland's
land cover dataset \citep{Chen.2015}, which contains ten land cover categories.
From the original layer we only retained the categories \textit{wetland} and
\textit{water bodies} and collectively reclassified them to \textit{water}.
Globeland had an original resolution of 30m x 30m, so we coarsened the layer to
250m x 250m using the mode of each 250m x 250m cell. Lastly, we improved river
representation using the MERIT Hydro dataset \citep{Yamazaki.2019}, from which
we added all rivers with a width of over 10m to our watermaps. The entire
process resulted in several hundred watermaps, of which 90 represented the
closest watermap to one of our GPS relocations.

%------------------------------------------------------------------------------
%	Appendix S4: Flood Pulse
%------------------------------------------------------------------------------
\newpage
\section{Typical Flood Pulse}

\begin{figure}[hbtp]
  \begin{center}
    \includegraphics[width = \textwidth]{99_FloodPulse.pdf}
    \caption{Sequence of flood maps showing a typical flood pulse throughout the
    year. The flood arrives from the north-western corner (so called
    ``pan-handle'') of the Okavango Delta and slowly descents through the delta
    in south-eastern direction, where it nourishes several tributaries. The
    extent of the flood peaks around August or September and then slowly
    retracts. Between December and March the reflectance properties of water and
    dryland change, which is why often no accurate flood maps can be obtained
    for these months using remote sensing techniques \citep{Wolski.2017}.}
    \label{FloodPulse}
  \end{center}
\end{figure}

%------------------------------------------------------------------------------
%  Appendix S5: Road Categories
%------------------------------------------------------------------------------
\newpage
\section{Road Categories}

\begin{table}[hbtp]
  \caption{Description of road types, as sourced from Open Street Map's mapping
  guide (\url{https://wiki.openstreetmap.org/wiki/Key:highway}). Roads types
  that were considered for the purpose of this study are shaded in light gray.}
  \label{RoadsDescription}
  \begin{center}
    \resizebox{\textwidth}{!}{
      \begin{tabular}{llp{12cm}}
      \toprule
      Group &
        Subgroup &
          Description \\
      \midrule
      \rowcolor[gray]{0.90}
      Roads &
        motorway &
          A restricted access major divided highway, normally with 2 or more
          running lanes plus emergency hard shoulder. Equivalent to the
          Freeway, Autobahn, etc. \\
      \rowcolor[gray]{0.90}
      Roads &
        trunk &
          The most important roads in a country's system that aren't
          motorways. Need not necessarily be a divided highway. \\
      \rowcolor[gray]{0.90}
      Roads &
        primary &
          The next most important roads in a country's system. Often link
          larger towns. \\
      \rowcolor[gray]{0.90}
      Roads &
        secondary &
          The next most important roads in a country's system. Often link
          towns. \\
      Roads &
        tertiary &
          The next most important roads in a country's system. Often link
          smaller towns and villages \\
      Roads &
        unclassified &
          The least important thorough roads in a country's system, i.e. minor
          roads of a lower classification than tertiary, but which serve a
          purpose other than access to properties. Often link villages and
          hamlets. \\
      Roads &
        residential &
          Roads which serve as an access to housing, without function of
          connecting settlements. Often lined with housing. \\
      Roads &
        service &
          For access roads to, or within an industrial estate, camp site,
          business park, car park etc. \\
      \rowcolor[gray]{0.90}
      Link roads &
        motorway\_link &
          The link roads (sliproads/ramps) leading to/from a motorway from/to
          a motorway or lower class highway. Normally with the same motorway
          restrictions. \\
      \rowcolor[gray]{0.90}
      Link roads &
        trunk\_link &
          The link roads (sliproads/ramps) leading to/from a trunk road
          from/to a trunk road or lower class highway. \\
      \rowcolor[gray]{0.90}
      Link roads &
        primary\_link &
          The link roads (sliproads/ramps) leading to/from a primary road
          from/to a primary road or lower class highway. \\
      \rowcolor[gray]{0.90}
      Link roads &
        secondary\_link &
          The link roads (sliproads/ramps) leading to/from a secondary road
          from/to a secondary road or lower class highway. \\
      Link roads &
        tertiary\_link &
          The link roads (sliproads/ramps) leading to/from a tertiary road
          from/to a tertiary road or lower class highway. \\
      Special road types &
        living\_street &
          For living streets, which are residential streets where pedestrians
          have legal priority over cars, speeds are kept very low and where
          children are allowed to play on the street. \\
      Special road types &
        pedestrian &
          For roads used mainly/exclusively for pedestrians in shopping and
          some residential areas which may allow access by motorised vehicles
          only for very limited periods of the day. \\
      Special road types &
        track &
          Roads for mostly agricultural or forestry uses. \\
      Special road types &
        bus\_guideway &
          A busway where the vehicle is guided by the way (though not a railway)
          and is not suitable for other traffic. \\
      Special road types &
        escape &
          For runaway truck ramps, runaway truck lanes, emergency escape
          ramps, or truck arrester beds. It enables vehicles with braking
          failure to safely stop. \\
      Special road types &
        raceway &
          A course or track for racing \\
      Special road types &
        road &
          A road/way/street/motorway/etc. of unknown type. It can stand for
          anything ranging from a footpath to a motorway. \\
      \bottomrule
      \end{tabular}
    }
  \end{center}
\end{table}

%------------------------------------------------------------------------------
%	Appendix S6: Human Influence
%------------------------------------------------------------------------------
\newpage
\section{Human Influence Proxy}
Because the layers \textit{human density}, \textit{farming}, and \textit{roads}
depicted features that were clustered in space, and because not all dispersing
coalitions moved within meaningful distance of all these features, we merged
them into a single general proxy rendering human influence
\Cref{HumanInfluence}. That is, we totaled values from the layers describing
\textit{human density} (continuous), \textit{farming} (binary), and
\textit{roads} (binary). This approach implied that \textit{roads} and
\textit{farms} entered the final layer with a value of 1, whereas human density
entered the final layer with a value \(\geq 0\) and potentially unbound. To
reduce the influence of outliers in human density estimates, totaled values were
limited to a maximum of 50, which visually resulted in a good balance between
high and low anthropogenic influence and was therefore considered appropriate
for our analysis. To render the fact that humans influence their surroundings
beyond their presence, we followed \cite{Elliot.2014} and applied to each
raster-cell a 5km focal buffer within which we summed up and log-transformed
human-influence values.

\begin{figure}[hbtp]
  \begin{center}
    \includegraphics[width = \textwidth]{99_HumanInfluence.pdf}
    \caption{Sequence of figures that exemplify how we combined the layers
    depicted in (a), (b), and (c) into a single layer for human influence (d).
    For better visibility we show the procedure only for the extent of the
    Okavango Delta. The layer in (a) is based on Facebook's high resolution
    human density dataset (\url{www.dataforgood.fb.com};
    \citealp{Facebook.2019}) and depicts the estimated number of humans living
    in each 250m x 250m raster-cell (coarsened from 30m x 30m). The layer in (b)
    is a binary layer and shows whether raster-cells are cover any sort of
    agricultural fields. Corresponding data was obtained through the Globeland
    and Cropland land cover datasets \citep{Chen.2015, Xiong.2017}. The layer in
    (c) shows the presence or absence of roads and is based on data from Open
    Street Map \citep{OpenStreetMap.2019}. We merged the layers in (a), (b), and
    (c) by summing up their values, truncating the summed values to a maximum of
    50. We then log-transforming the values and applied to each raster cell a
    focal buffer of 5km within which we totaled human influence values. The
    layer in (d) depicts the final human influence layer that entered our
    habitat selection model.}
    \label{HumanInfluence}
  \end{center}
\end{figure}

%------------------------------------------------------------------------------
%	Appendix S7: Covariates
%------------------------------------------------------------------------------
\newpage
\section{Covariates \& Sources}

\begin{table}[hbtp]
  \begin{center}
    \caption{Overview of spatial covariates and their sources. We extracted
    covariates along realized and random steps. For continuous covariates we
    calculated average values along steps, for categorical covariates the
    percentage coverage along steps. We also prepared two covariates indicating
    the distance to water and distance to roads, respectively. We square-rooted
    the values for these two covariates to render a decreasing marginal impact
    of the effect of distance. Finally, we derived a binary indicator of whether
    a step crossed a road or not.}
    \label{Appendix:Sources}
    \resizebox{\textwidth}{!} {
      \begin{threeparttable}
        \begin{tabular}{lllll}
        \hline
        Category &
          Covariate &
            Description &
              Values &
                Source \\
        \midrule
        \multirow{5}{*}{Land Cover}
          & Water
            & Percentage cover of water
              & 0-100\%
                & (1) (2) (3) \\
          & Dryland*
            & Percentage cover of dryland
              & 0-100\%
                & (1) (2) (3) \\
          & DistanceToWater
            & Average distance to nearest water source
              & \(\geq\) 0m
                & (1) (2) (3) \\
          & Shrubs/Grassland
            & Average non-tree vegetation
              & 0-100\%
                & (4) \\
          & Trees
            & Average tree-vegetation
              & 0-100\%
                & (4) \\
          & Bareland*
            & Average non-vegetated area
              & 0-100\%
                & (4) \\
        \hdashline
        \multirow{2}{*}{Protection Status}
          & Protected
            & Percentage cover of protected area
              & 0-100\%
                & (5) \\
          & Pastoral*
            & Percentage cover of unprotected area
              & 0-100\%
                & (5) \\
        \hdashline
        \multirow{3}{*}{Human Influence}
          & Human Influence
            & Average human influence
              & \(\geq\) 0
                & (1) (6) (7) (8)\\
          & DistanceToRoads
            & Average distance to nearest road
              & \(\geq\) 0m
                & (1) (6) (7) (8)\\
          & RoadCrossing
            & Binary; whether a step crossed a road
              & 0, 1
                & (1) (6) (7) (8)\\
        \hline
        \end{tabular}
        \begin{tablenotes}
          \item \textit{Sources:} (1) \cite{Chen.2015} (2) \cite{Schaaf.2015} (3)
          \cite{Yamazaki.2019} (4) \cite{Dimiceli.2015} (5) \cite{PeaceParks.2019}
          (6) \cite{Facebook.2019} (7) \cite{OpenStreetMap.2019} (8)
          \cite{Xiong.2017}
          \item
          \item \textit{* Note:} The covariates \textit{Water} and
          \textit{Dryland} added up to 100\%, which is why only \textit{Water} was
          included as explanatory variable in our models. The same applied for the
          group \textit{Shrubs/Grassland}, \textit{Trees}, and \textit{Bareland},
          where we omitted \textit{Bareland} for modeling. Finally, from the group
          \textit{Protected} and \textit{Pastoral}, we only included
          \textit{Protected} in our models.
        \end{tablenotes}
      \end{threeparttable}
    }
  \end{center}
\end{table}

\newpage
\section{Integrated Step Selection Function}
We used an integrated step selection function (iSSF; \citealp{Avgar.2016}) to
investigate dispersers' selection or avoidance of spatial covariates. In the
iSSF framework, covariates experienced along realized steps are contrasted with
covariates experienced along alternative random steps that the animal could have
taken but decided not to. A step in this framework is defined as the connecting
line between two consecutive GPS relocations \citep{Turchin.1998}. In contrast
to regular SSFs, iSSFs require to include movement metrics as covariates in the
corresponding conditional logistic regression model. Their inclusion, in turn,
allows simultaneous inference on habitat and movement preferences, as well as to
reduce potential biases in estimated habitat preferences \citep{Forester.2009,
Warton.2013, Avgar.2016}.

To conduct iSSF analysis, we followed the recommendations described in Appendix
S1 of the publication by \cite{Avgar.2016}. We prepared our GPS relocation data
for iSSF-analysis using the R-package \textit{amt} \citep{Amt.2019} and coerced
relocations recorded during dispersal to steps that were regularly spaced four
hours apart. Steps that were separated by more than four hours (e.g. due to GPS
failure) were omitted from further analysis (allowing for a minor mismatch of up
to 15 minutes). Each remaining step was paired with 24 random steps, generated
by sampling turning angles from a uniform distribution U(\(-\pi, \pi\)) and step
lengths from a gamma distribution that was fitted using realized step lengths
(shape = 0.3677, scale = 6'302). Together, a realized and its 24 associated
random steps formed a stratum of 25 steps that received a unique identifier.

We extracted spatial covariates along realized and random steps. For continuous
covariates, we calculated the average value, for categorical covariates the
percentage cover along the step. We further derived a binary variable indicating
whether or not a step crossed a road and we identified the average distance of
each step to the nearest source of water or the nearest road. We square-rooted
these values to render a decreasing marginal impact of distance. We standardized
continuous covariates to a mean of zero and a standard deviation of one using a
z-score transformation. Furthermore, we screened covariates for correlation
using Pearson's Correlation Coefficient. None of the covariates were correlated
(\(|r| > 0.6\); \citealp{Latham.2011}) and we retained all of them for modeling.
In our regression model we included two additional covariates, namely the cosine
of the turning angle (\(cos(ta)\)) and the logarithm of the step length
(\(log(sl)\)) \citep{Avgar.2016}. The movement metric \(cos(ta)\) serves to
describe the directionality of a step, as it transforms the circular measure of
(\(-\pi\) to \(\pi\)) into a linear measure (-1, 1). Thus, positive values
indicate forward movements, whereas negative values indicated backward movements
\citep{Turchin.1998}. The movement metric \(log(sl)\), on the other hand, is as
an indicator of the preferred step length. Since in our case steps were spaced
by four hours, \(log(sl)\) can also be interpreted as movement rate.

We then used the iSSF framework and parameterized a habitat selection model that
further served to predict landscape permeability. This habitat selection model
operated under the assumption that dispersing wild dogs assigned a selection
score \(w(x)\) of the following exponential form to each realized and random
step:

\begin{equation}
\label{EQ2}
  w(x) = exp(\beta_1 x_1 + \beta_2 x_2 + ... + \beta_n x_n)
\end{equation}

The selection score \(w(x)\) of a step depended on its associated covariates
(\(x_1, x_2, ..., x_n\)), as well as on the animal's preferences for these
covariates (\(\beta_1, \beta_2, ..., \beta_n\)). The probability that a step
\(i\) was realized \(P(Y_{i} = 1\)) was then contingent on the step's selection
score, as well as on the selection scores of all alternative steps in the
stratum:

\begin{equation}
\label{EQ3}
  P(Y_{i} = 1 | Y_{1} + Y_{2} + ... + Y_{i} = 1) =
  \frac{w(x_{i})}{w(x_{1}) + w(x_{2}) + ... + w(x_{i})}
\end{equation}

\noindent Habitat and movement preferences of interest, i.e. the \(\beta\)'s,
were estimated by comparing realized (scored 1) and random (scored 0) steps in a
conditional logistic regression model \citep{Fortin.2005}. In this model,
positive \(\beta\)-coefficients indicate selection of a covariate, negative
\(\beta\)-coefficients avoidance of a covariate. To deal with multiple
individuals, we applied mixed effects conditional logistic regression analysis
following \cite{Muff.2020}. We implemented their method using the R-package
\textit{glmmTMB} \citep{Mollie.2017} and used dispersal coalition ID to model
random intercepts and slopes.

We defined the movement metrics \(cos(ta)\) and \(log(sl)\) as core covariates
and ran forward model selection based on Akaike's Information Criterion (AIC;
\citealp{Burnham.2002} for all other covariates. We ranked models according to
AIC, assessed relative model weights, and identified the most parsimonious
model. Due to convergence issues, we were unable to model interactions between
covariates.

To validate the predictive power of the most parsimonious selection model, we
ran k-fold cross-validation for case-control studies as described in
\cite{Fortin.2009}. Using 80\% of randomly selected strata, we parameterized a
selection model and predicted selection scores \(w(x)\) for all steps in the
remaining 20\% of strata. According to predicted selection scores we assigned
ranks 1-25 within each stratum, with rank 1 indicating the highest selection
score. We identified the realized step's rank in each stratum and tallied rank
frequencies of realized steps across all strata. Finally, we carried out a
Spearman-rank correlation analysis between ranks and associated frequencies and
we recorded the correlation coefficient (\(r_{s, realized}\)). We repeated this
procedure 100 times with replacement and computed the mean correlation
coefficient (\(\bar{r}_{s, realized}\)), as well as its 95\% confidence
interval. For comparison, we also repeated the same procedure 100 times assuming
completely randomized preferences. We implemented randomized preferences by
omitting the realized step from each stratum and identifying the rank of a
randomly chosen random step within each stratum (now only ranks 1-24). Again, we
calculated Spearman's rank correlation coefficient (\(r_{s, random}\)), its mean
across repetitions (\(\bar{r}_{s, random}\)), and its 95\% confidence interval.
Ultimately, the validation proofed a significant prediction in case the
confidence intervals of \(\bar{r}_{s, realized}\) and \(\bar{r}_{s, random}\)
did not overlap.

\newpage
\section{Identification of Least-Cost Paths \& Corridors}
We implemented factorial LCP analysis between source points using the R-package
\textit{gdistance} (Figure S.7; \citealp{vanEtten.2017}). The package translated
the (unscaled) permeability surface into a network of nodes to find shortest
effective distances between source points based on probabilities of moving from
cell to cell. In our case, the transition probability of moving between two
adjacent cells depended on their averaged permeability. We allowed individuals
to move from each cell to the cell's eight surrounding neighbors (i.e. Moores
neighborhood) and applied a geographic correction to account for the fact that
diagonal neighbors were more remote than orthogonal neighbors. Because African
wild dogs have been observed to cover large dispersal distances
\citep{DaviesMostert.2012, Masenga.2016, Cozzi.2020}, we did not limit LCPs to a
maximal effective cost. After computation, we tallied overlapping LCPs and
identified high-frequency routes.

We also calculated factorial LCCs \citep{Pinto.2009, Sawyer.2011, Elliot.2014},
again using the R-package \textit{gdistance} (Figure S.7;
\citealp{vanEtten.2017}). To identify LCCs, we first computed for each source
point a cumulative cost map, which indicated the total minimal costs required to
get from the source point to any other location in the study area. We then
obtained an LCC between two source points by adding up their cumulative cost
maps and masking out all cell-values exceeding the lowest cell-value by more
than 5\% \citep{Pinto.2009}. We repeated this procedure for each possible unique
pairwise combination of source points and thereby identified LCCs between all 68
selected source points. We normalized the resulting corridor-maps to range from
zero to one and tallied them into a single connectivity map.

\begin{figure}[hbtp]
  \begin{center}
  \begin{minipage}{.32\linewidth}
    \begin{subfigure}[t]{\linewidth}
        \includegraphics[width=\textwidth]{99_LeastCostPaths(Example1).pdf}
    \end{subfigure}
  \end{minipage}
  \begin{minipage}{.32\linewidth}
    \begin{subfigure}[t]{\linewidth}
        \includegraphics[width=\textwidth]{99_LeastCostPaths(Example2).pdf}
    \end{subfigure}
  \end{minipage}
  \begin{minipage}{.32\linewidth}
    \begin{subfigure}[t]{\linewidth}
        \includegraphics[width=\textwidth]{99_LeastCostPaths(Example3).pdf}
    \end{subfigure}
  \end{minipage}
  \caption{Images illustrating the process of identifying a least-cost corridor
  between source points A and B following \cite{Pinto.2009}. (a) Example of a
  permeability surface, which determines the costs of movement. (b1) Cumulative
  cost map for point A, depicting the total minimal costs necessary to get from
  point A to every other location. (b2) Cumulative cost map for point B,
  depicting the total minimal costs to get from point B to every other location.
  (c) Summed cost maps of points A and B. (d) Masked out corridor containing
  pixels that do not exceed the cheapest pixel by more than 5\%.}
  \label{LCCExample}
  \end{center}
\end{figure}

\newpage
\section{Model Selection Results}

\begin{table}[hbpt]
  \caption{Results from the forward model selection procedure based on Akaike's
  Information Criterion (AIC; \citealp{Burnham.2002}) for the habitat selection
  model. The most parsimonious model outperformed all other models (\(\Delta AIC
  > \) 2) and received a weight of one.}
  \label{ModelAICs}
  \begin{center}
    \resizebox{\textwidth}{!}{
      \begin{threeparttable}
        \begin{tabular}{lllll}
        \toprule
          Covariates & AIC & \(\Delta\)AIC & Weight & LogLik \\
          \midrule
          cos(ta) + log(sl) + W + T + DTW + HB5 + S & 90091.40 & 0.00 & 1.00 & -45030.70 \\
          cos(ta) + log(sl) + W + T + DTW + HB5 + S + DTR & 90093.51 & 2.11 & 0.00 & -45029.75 \\
          cos(ta) + log(sl) + W + T + DTW + HB5 & 90094.79 & 3.38 & 0.00 & -45034.39 \\
          cos(ta) + log(sl) + W + T + DTW + HB5 + S + P & 90095.05 & 3.65 & 0.00 & -45030.52 \\
          cos(ta) + log(sl) + W + T + DTW + S & 90096.95 & 5.55 & 0.00 & -45035.47 \\
          cos(ta) + log(sl) + W + T + DTW + HB5 + DTR & 90097.02 & 5.62 & 0.00 & -45033.51 \\
          cos(ta) + log(sl) + W + T + DTW + HB5 + S + DTR + P & 90097.38 & 5.98 & 0.00 & -45029.69 \\
          cos(ta) + log(sl) + W + T + DTW + HB5 + S + RC & 90097.53 & 6.13 & 0.00 & -45029.76 \\
          cos(ta) + log(sl) + W + T + DTW + HB5 + P & 90098.46 & 7.06 & 0.00 & -45034.23 \\
          cos(ta) + log(sl) + W + T + DTW + HB5 + S + DTR + RC & 90099.62 & 8.22 & 0.00 & -45028.81 \\
          cos(ta) + log(sl) + W + T + DTW & 90099.97 & 8.57 & 0.00 & -45038.99 \\
          cos(ta) + log(sl) + W + T + DTW + HB5 + RC & 90100.68 & 9.28 & 0.00 & -45033.34 \\
          cos(ta) + log(sl) + W + T + DTW + DTR & 90101.68 & 10.28 & 0.00 & -45037.84 \\
          cos(ta) + log(sl) + W + T + DTW + P & 90103.07 & 11.67 & 0.00 & -45038.54 \\
          cos(ta) + log(sl) + W + T + DTW + HB5 + S + DTR + P + RC & 90103.48 & 12.08 & 0.00 & -45028.74 \\
          cos(ta) + log(sl) + W + T + DTW + RC & 90104.72 & 13.32 & 0.00 & -45037.36 \\
          cos(ta) + log(sl) + W + T + HB5 & 90123.71 & 32.31 & 0.00 & -45050.85 \\
          cos(ta) + log(sl) + W + T & 90128.17 & 36.77 & 0.00 & -45055.09 \\
          cos(ta) + log(sl) + W + T + S & 90129.36 & 37.96 & 0.00 & -45053.68 \\
          cos(ta) + log(sl) + W + T + DTR & 90130.11 & 38.71 & 0.00 & -45054.05 \\
          cos(ta) + log(sl) + W + T + P & 90131.75 & 40.35 & 0.00 & -45054.88 \\
          cos(ta) + log(sl) + W + T + RC & 90133.53 & 42.13 & 0.00 & -45053.77 \\
          cos(ta) + log(sl) + W + DTW & 90138.92 & 47.52 & 0.00 & -45060.46 \\
          cos(ta) + log(sl) + W + HB5 & 90149.30 & 57.90 & 0.00 & -45065.65 \\
          cos(ta) + log(sl) + W + S & 90151.14 & 59.74 & 0.00 & -45066.57 \\
          cos(ta) + log(sl) + W & 90153.70 & 62.30 & 0.00 & -45069.85 \\
          cos(ta) + log(sl) + W + DTR & 90156.32 & 64.92 & 0.00 & -45069.16 \\
          cos(ta) + log(sl) + W + P & 90157.23 & 65.83 & 0.00 & -45069.62 \\
          cos(ta) + log(sl) + W + RC & 90159.32 & 67.91 & 0.00 & -45068.66 \\
          cos(ta) + log(sl) + S & 90161.07 & 69.67 & 0.00 & -45073.54 \\
          cos(ta) + log(sl) + DTW & 90262.03 & 170.63 & 0.00 & -45124.01 \\
          cos(ta) + log(sl) + T & 90301.56 & 210.16 & 0.00 & -45143.78 \\
          cos(ta) + log(sl) + HB5 & 90307.55 & 216.15 & 0.00 & -45146.78 \\
          cos(ta) + log(sl) + DTR & 90316.37 & 224.97 & 0.00 & -45151.19 \\
          cos(ta) + log(sl) + P & 90316.62 & 225.22 & 0.00 & -45151.31 \\
          cos(ta) + log(sl) + RC & 90318.46 & 227.06 & 0.00 & -45150.23 \\
         \bottomrule
       \end{tabular}
       \begin{tablenotes}
         \item \textit{Note:} W = Water, DTW = Distance To Water, S =
         Shrubs/Grassland, T = Trees, P = Protected, HI = Human Influence, RC =
         Road Crossing, DTR = Distance To Roads.
       \end{tablenotes}
     \end{threeparttable}
    }
  \end{center}
\end{table}

\newpage
\begingroup
\singlespacing
\bibliography{Literatur}
\endgroup

\end{document}
