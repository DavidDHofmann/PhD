% latex table generated in R 3.6.3 by xtable 1.8-4 package
% Tue Apr 21 10:00:46 2020
\begin{tabular}{lll}
  \toprule
Group & Subgroup & Description \\ 
  \midrule
Roads & motorway & A restricted access major divided highway, normally with 2 or more running lanes plus emergency hard shoulder. Equivalent to the Freeway, Autobahn, etc.. \\ 
  Roads & trunk & The most important roads in a country's system that aren't motorways. (Need not necessarily be a divided highway.) \\ 
  Roads & primary & The next most important roads in a country's system. (Often link larger towns.) \\ 
  Roads & secondary & The next most important roads in a country's system. (Often link towns.) \\ 
  Roads & tertiary & The next most important roads in a country's system. (Often link smaller towns and villages) \\ 
  Roads & unclassified & The least important through roads in a country's system – i.e. minor roads of a lower classification than tertiary, but which serve a purpose other than access to properties. (Often link villages and hamlets.) \\ 
  Roads & residential & Roads which serve as an access to housing, without function of connecting settlements. Often lined with housing. \\ 
  Roads & service & For access roads to, or within an industrial estate, camp site, business park, car park etc. Can be used in conjunction with service=* to indicate the type of usage and with access=* to indicate who can use it and in what circumstances. \\ 
  Link roads & motorway\_link & The link roads (sliproads/ramps) leading to/from a motorway from/to a motorway or lower class highway. Normally with the same motorway restrictions. \\ 
  Link roads & trunk\_link & The link roads (sliproads/ramps) leading to/from a trunk road from/to a trunk road or lower class highway. \\ 
  Link roads & primary\_link & The link roads (sliproads/ramps) leading to/from a primary road from/to a primary road or lower class highway. \\ 
  Link roads & secondary\_link & The link roads (sliproads/ramps) leading to/from a secondary road from/to a secondary road or lower class highway. \\ 
  Link roads & tertiary\_link & The link roads (sliproads/ramps) leading to/from a tertiary road from/to a tertiary road or lower class highway. \\ 
  Special road types & living\_street & For living streets, which are residential streets where pedestrians have legal priority over cars, speeds are kept very low and where children are allowed to play on the street. \\ 
  Special road types & pedestrian & For roads used mainly/exclusively for pedestrians in shopping and some residential areas which may allow access by motorised vehicles only for very limited periods of the day. To create a 'square' or 'plaza' create a closed way and tag as pedestrian and also with area=yes. \\ 
  Special road types & track & Roads for mostly agricultural or forestry uses. To describe the quality of a track, see tracktype=*. Note: Although tracks are often rough with unpaved surfaces, this tag is not describing the quality of a road but its use. Consequently, if you want to tag a general use road, use one of the general highway values instead of track. \\ 
  Special road types & bus\_guideway & A busway where the vehicle guided by the way (though not a railway) and is not suitable for other traffic. Please note: this is not a normal bus lane, use access=no, psv=yes instead! \\ 
  Special road types & escape & For runaway truck ramps, runaway truck lanes, emergency escape ramps, or truck arrester beds. It enables vehicles with braking failure to safely stop. \\ 
  Special road types & raceway & A course or track for (motor) racing \\ 
  Special road types & road & A road/way/street/motorway/etc. of unknown type. It can stand for anything ranging from a footpath to a motorway. This tag should only be used temporarily until the road/way/etc. has been properly surveyed. If you do know the road type, do not use this value, instead use one of the more specific highway=* values. \\ 
  Paths & footway & For designated footpaths; i.e., mainly/exclusively for pedestrians. This includes walking tracks and gravel paths. If bicycles are allowed as well, you can indicate this by adding a bicycle=yestag. Should not be used for paths where the primary or intended usage is unknown. Use highway=pedestrian for pedestrianised roads in shopping or residential areas and highway=track if it is usable by agricultural or similar vehicles. \\ 
  Paths & bridleway & For horse riders. Equivalent to highway=path + horse=designated. \\ 
  Paths & steps & For flights of steps (stairs) on footways. Use with step\_count=* to indicate the number of steps \\ 
  Paths & path & A non-specific path. Use highway=footway for paths mainly for walkers, highway=cycleway for one also usable by cyclists, highway=bridlewayfor ones available to horse riders as well as walkers and highway=track for ones which is passable by agriculture or similar vehicles. \\ 
  Cycleway & lane & A lane is a route that lies within the roadway \\ 
  Cycleway & opposite & Used on ways with oneway=yes where it is legally permitted to cycle in both directions. Used together with oneway:bicycle=no. \\ 
  Cycleway & opposite\_lane & Used on ways with oneway=yes that have a cycling lane going the opposite direction of normal traffic flow (a "contraflow" lane). Used together with oneway:bicycle=no. \\ 
  Cycleway & track & A track provides a route that is separated from traffic. In the United States, this term is often used to refer to bike lanes that are separated from lanes for cars by pavement buffers, bollards, parking lanes, and curbs. Note that a cycle track may alternatively be drawn as a separate way next to the road which is tagged as highway=cycleway. \\ 
  Cycleway & opposite\_track & Used on ways with oneway=yes that have a cycling track going the opposite direction of normal traffic flow. Used together with oneway:bicycle=no. \\ 
  Cycleway & share\_busway & There is a bus lane that cyclists are permitted to use. \\ 
  Cycleway & opposite\_share\_busway & Used on ways with oneway=yes that have a bus lane that cyclists are also permitted to use, and which go in the opposite direction to normal traffic flow (a "contraflow" bus lane). Used together with oneway:bicycle=no. \\ 
  Cycleway & shared\_lane & Cyclists share a lane with motor vehicles, but there are markings indicating that they should share the lane with motorists. In some places these markings are known as "sharrows" ('sharing arrows') and this is the tag to use for those. \\ 
  Busway & lane & Bus lane on both sides of the road. \\ 
   \bottomrule
\end{tabular}
