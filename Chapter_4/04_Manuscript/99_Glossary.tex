\begin{table}

\caption{\label{tab:Glossary}Glossary of terms. Terms in the glossary are printed in bold at first occurrence in the main text. Definitions are always given in the context of step selection functions (SSFs).}
\centering
\resizebox{\linewidth}{!}{
\begin{tabular}[t]{>{}l>{\raggedright\arraybackslash}p{10 cm}}
\toprule
Term & Definition\\
\midrule
\textbf{\cellcolor{gray!6}{Animal locations}} & \cellcolor{gray!6}{A series of telemetry data points that include date, time, longitude, and latitude information, describing when and where an animal was observed or recorded.}\\
\textbf{Step} & A straight line connecting two consecutive locations.\\
\textbf{\cellcolor{gray!6}{Observed step}} & \cellcolor{gray!6}{A step that connects two observed animal locations.}\\
\textbf{Random step} & A step that connects an observed animal location with a random location. Random locations are typically generated by combining an observed animal location with random step lengths and turning angles.\\
\textbf{\cellcolor{gray!6}{Step length}} & \cellcolor{gray!6}{The Euclidean distance of a step.}\\
\addlinespace
\textbf{Turning angle} & A measure of the change in direction between two consecutive steps.\\
\textbf{\cellcolor{gray!6}{Habitat-selection function}} & \cellcolor{gray!6}{A probabilistic description of an animal's habitat preferences. It describes how an animal selects habitat when not constrained by its movement capacity. Also referred to as movement-free habitat-selection function.}\\
\textbf{Movement kernel} & A probabilistic description of an animal's movement capacity. It describes how an animal would move when not constrained by habitat selection. Also referred to as selection-free movement kernel.\\
\textbf{\cellcolor{gray!6}{Trajectory}} & \cellcolor{gray!6}{A sequence of animal locations collected on the same individual.}\\
\textbf{Step duration} & The time interval associated with a particular step, i.e., the time elapsed between two consecutive animal locations.\\
\addlinespace
\textbf{\cellcolor{gray!6}{Regular animal locations}} & \cellcolor{gray!6}{A series of animal locations that have been obtained at regularly spaced time intervals, such as every hour.}\\
\textbf{Irregular animal locations} & A series of animal locations collected at irregular time intervals.\\
\textbf{\cellcolor{gray!6}{Regular step durations}} & \cellcolor{gray!6}{Step durations that occur when animal locations are successfully collected at regular time intervals.}\\
\textbf{Irregular step durations} & Step durations that occur when animal locations are not successfully collected at regular time intervals.\\
\textbf{\cellcolor{gray!6}{Missingness}} & \cellcolor{gray!6}{The fraction of animal locations that should have been collected but, for some reason, were not. For example, if only eight out of ten expected animal locations were successfully collected, the missingness would be 0.2 (i.e., 20\%).}\\
\addlinespace
\textbf{Forgiveness} & The maximum step duration, measured in multiples of the regular step duration, a modeler is willing to include in the step-selection analysis. A modeler with a forgiveness of one, for instance, only considers regular steps, while a modeler with a forgiveness of two would consider irregular steps up to twice the regular step duration.\\
\textbf{\cellcolor{gray!6}{Burst}} & \cellcolor{gray!6}{A sequence of consecutive animal locations equally spaced in time and with steps were the step duration does not exceed the forgiveness.}\\
\textbf{Valid step} & A step for which a step length, turning angle, and step duration can be computed, and for which the step duration does not exceed the forgiveness of the modeler. These steps can be used for step-selection analysis.\\
\bottomrule
\end{tabular}}
\end{table}
